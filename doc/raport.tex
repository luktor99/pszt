\documentclass[11pt,a4paper]{article}
\usepackage[OT4]{polski}
\usepackage[utf8]{inputenc}
\usepackage[inner=2.5cm,outer=2.5cm, tmargin=2.5cm,bmargin=2.5cm]{geometry}
\usepackage{amsmath}
\usepackage{relsize,amsfonts}
\usepackage{enumitem}
\usepackage{graphicx}

\newcommand\bigexists{%
  \mathop{\lower0.75ex\hbox{\ensuremath{%
    \mathlarger{\mathlarger{\mathlarger{\mathlarger{\exists}}}}}}}%
  \limits}
  
\newcommand\bigforall{%
  \mathop{\lower0.75ex\hbox{\ensuremath{%
    \mathlarger{\mathlarger{\mathlarger{\mathlarger{\forall}}}}}}}%
  \limits}
  
\title{Podstawy Sztucznej Inteligencji\\\large \medskip Projekt: WG.AE.1\\}
\author{Marcin Baran numer indeksu\\ Łukasz Kilaszewski numer indeksu\\ Mateusz Perciński Z59827}
\date{13 czerwca 2017}

\begin{document}

\maketitle
\tableofcontents

\section*{Treść zadania}
\textbf{WG.AE1 Rozwożenie mebli} \\
Zaplanować trasę samochodu ciężarowego rozwożącego meble. Każdy mebel ma określoną
wagę oraz miasto przeznaczenia. Zużycie paliwa przez samochód ciężarowy jest zależne od
masy przewożonego ładunku. Zaplanować trasą rozwiezienia mebli która jest optymalna ze
względu na zużycie paliwa. Program powinien na bieżąco prezentować jakość znalezionego
rozwiązania w funkcji numeru pokolenia.
\section*{Podział pracy}
Marcin Baran - Definicja zadania\\
Łukasz Kilaszewski - Implementacja programu\\
Mateusz Perciński - Raport oraz Wnioski

\section{Definicja problemu}
zestawienie kluczowych decyzji projektowych,

\section{Implementacja}

\subsection{Struktura programu}
opis struktury
programu
\subsection{Instrukcja dla użytkownika}

\section{Testy i osiągnięte rezultaty}

\section{Wnioski}
wnioski dotyczące osiągniętych rezultatów. 

Zaplanować trasę samochodu ciężarowego rozwożącego meble. Każdy mebel ma określoną
wagę oraz miasto przeznaczenia. Zużycie paliwa przez samochód ciężarowy jest zależne od
masy przewożonego ładunku. Zaplanować trasą rozwiezienia mebli która jest optymalna ze
względu na zużycie paliwa. Program powinien na bieżąco prezentować jakość znalezionego
rozwiązania w funkcji numeru pokolenia.
Zaplanować trasę samochodu ciężarowego rozwożącego meble. Każdy mebel ma określoną
wagę oraz miasto przeznaczenia. Zużycie paliwa przez samochód ciężarowy jest zależne od
masy przewożonego ładunku. Zaplanować trasą rozwiezienia mebli która jest optymalna ze
względu na zużycie paliwa. Program powinien na bieżąco prezentować jakość znalezionego
rozwiązania w funkcji numeru pokolenia.
Zaplanować trasę samochodu ciężarowego rozwożącego meble. Każdy mebel ma określoną
wagę oraz miasto przeznaczenia. Zużycie paliwa przez samochód ciężarowy jest zależne od
masy przewożonego ładunku. Zaplanować trasą rozwiezienia mebli która jest optymalna ze
względu na zużycie paliwa. Program powinien na bieżąco prezentować jakość znalezionego
rozwiązania w funkcji numeru pokolenia.
Zaplanować trasę samochodu ciężarowego rozwożącego meble. Każdy mebel ma określoną
wagę oraz miasto przeznaczenia. Zużycie paliwa przez samochód ciężarowy jest zależne od
masy przewożonego ładunku. Zaplanować trasą rozwiezienia mebli która jest optymalna ze
względu na zużycie paliwa. Program powinien na bieżąco prezentować jakość znalezionego
rozwiązania w funkcji numeru pokolenia.
Zaplanować trasę samochodu ciężarowego rozwożącego meble. Każdy mebel ma określoną
wagę oraz miasto przeznaczenia. Zużycie paliwa przez samochód ciężarowy jest zależne od
masy przewożonego ładunku. Zaplanować trasą rozwiezienia mebli która jest optymalna ze
względu na zużycie paliwa. Program powinien na bieżąco prezentować jakość znalezionego
rozwiązania w funkcji numeru pokolenia.
Zaplanować trasę samochodu ciężarowego rozwożącego meble. Każdy mebel ma określoną
wagę oraz miasto przeznaczenia. Zużycie paliwa przez samochód ciężarowy jest zależne od
masy przewożonego ładunku. Zaplanować trasą rozwiezienia mebli która jest optymalna ze
względu na zużycie paliwa. Program powinien na bieżąco prezentować jakość znalezionego
rozwiązania w funkcji numeru pokolenia.
Zaplanować trasę samochodu ciężarowego rozwożącego meble. Każdy mebel ma określoną
wagę oraz miasto przeznaczenia. Zużycie paliwa przez samochód ciężarowy jest zależne od
masy przewożonego ładunku. Zaplanować trasą rozwiezienia mebli która jest optymalna ze
względu na zużycie paliwa. Program powinien na bieżąco prezentować jakość znalezionego
rozwiązania w funkcji numeru pokolenia.
Zaplanować trasę samochodu ciężarowego rozwożącego meble. Każdy mebel ma określoną
wagę oraz miasto przeznaczenia. Zużycie paliwa przez samochód ciężarowy jest zależne od
masy przewożonego ładunku. Zaplanować trasą rozwiezienia mebli która jest optymalna ze
względu na zużycie paliwa. Program powinien na bieżąco prezentować jakość znalezionego
rozwiązania w funkcji numeru pokolenia.

\end{document}